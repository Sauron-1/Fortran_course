\documentclass{ctexart}

\usepackage{geometry}
\geometry{a4paper, centering, scale=0.8}
\usepackage{listings}
\usepackage{fontspec}
\usepackage{xcolor}
\lstset{
    basicstyle=\fontspec{Courier New Bold},
    columns=fixed,
    numbers=left,
    frame=none,
    backgroundcolor=\color[RGB]{200,200,200},
    keywordstyle=\color[RGB]{40,40,255},
    numberstyle=\footnotesize\color{darkgray},
    commentstyle=\it\color[RGB]{0,96,96},
    stringstyle=\rmfamily\slshape\color[RGB]{128,0,0}\fontspec{Courier New Bold},
    showstringspaces=false,
    language=fortran,
}


\homework{六}

\begin{document}

\maketitle

\begin{answer}

    \question{1}
    给出函数表
    \begin{tabular}{c|cccc}
        \hline
        x & 1.05 & 1.10 & 1.15 & 1.20 \\
        \hline
        f(x) & 2.12 & 2.20 & 2.17 & 2.32\\
        \hline
    \end{tabular}
    构造拉格朗日插值函数,计算f(1.075)和f(1.175)的近似值。\\

    直接套公式计算即可。其中乘法用product函数实现。已部分并行化。代码如下:
    \code{1.f90}
    输出文件为\file{output.txt}。

    \question{2}
    IDL编写拉格朗日插值,自己写/调包各一个,并比较。\\

    买不起IDL,支持正版。用python完成。\\
    同上,直接套公式。为复用性,定义为类。代码如下:
    \codes{python}{lagrange.py}
    输出图片:
    \pict{lagrange.eps}\\
    调用scipy.interpolate.interp1d完成,代码如下:
    \codes{python}{scipy-interpolate.py}
    输出图片:
    \pict{scipy-interpolate.eps}\\
    结果一样(为什么不呢?)。

    \question{3}
    求高次多项式的全部实根。\\

    考虑用矩阵实现。由于n阶方阵有n个特征值,可构造方阵,使其特征值为n次方程的n个根。
    考虑如下矩阵:
    \[ M = \begin{pmatrix}
        0 & 0 & \dots & -p_0 \\
        1 & 0 & \dots & -p_1 \\
        \vdots & \ddots & & \vdots \\
        0 & \dots & 1 & -p_{n-1}
    \end{pmatrix} \]
    式中,$p_i$为多项式中$x^i$项系数。容易算出,$M$的特征多项式为:
    \[p_0 + p_1\lambda + p_2\lambda^2 + \dots + p_{n-1}\lambda^{n-1} + \lambda^{n}\]
    所以$M$的特征值即为多项式方程的解。

    例程中使用QR方法求解特征值,算法参考自\href{https://zh.wikipedia.org/wiki/QR\%E5\%88\%86\%E8\%A7\%A3}{维基百科}。
    经过足够的迭代后,矩阵$R\times Q$对角线上的元素即为矩阵的实特征值,如果该元素下方即左方元素为零的话。
    而其所有复特征值将在结果中表现为对角线附近$2\times 2$的方阵的特征值。
    参考\href{http://people.inf.ethz.ch/arbenz/ewp/Lnotes/chapter4.pdf}{http://people.inf.ethz.ch/arbenz/ewp/Lnotes/chapter4.pdf}

    在实验中,存在部分使得QR算法不收敛的情况,此时该算法将给出错误的结果。这种情况很少见。
    算法思路参考自\href{https://github.com/numpy/numpy}{numpy}源码。

    求解代码如下:
    \code{util.f90}
    测试代码如下:
    \code{test.f90}
    测试输入文件为\file{input.txt},输出文件为\file{output.txt},输入提示保存为\file{err.txt}。

    \question{4}
    用四阶龙格-库塔格式求解初值问题
    \[ \frac{dy}{dx}=y^2\cos x\]
    \[y(0) = 1\]
    \[0 \le x \le 0.8\] \\

    直接套用公式求解即可。代码如下:
    \code{runge-kutta.f90}
    测试代码如下:
    \code{test.f90}
    输出文件为\file{res.dat}。\\
    画图代码如下:
    \codes{python}{plot.py}
    图像:
    \pict{res.eps} 

\end{answer}

\end{document}
