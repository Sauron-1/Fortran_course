\documentclass{ctexart}
\usepackage{geometry}
\geometry{a4paper, centering, scale=0.8}
\usepackage{listings}
\usepackage{fontspec}
\usepackage{xcolor}
\usepackage{enumerate}
\usepackage{graphicx}
\usepackage{amsmath}
\usepackage{authblk}
\usepackage[colorlinks=true,filecolor=blue,urlcolor=blue,linkcolor=blue]{hyperref}
\lstset{
    basicstyle=\fontspec{Courier Bold},
    columns=fixed,
    numbers=left,
    frame=none,
    backgroundcolor=\color[RGB]{200,200,200},
    keywordstyle=\color[RGB]{40,40,255},
    numberstyle=\footnotesize\color{darkgray},
    commentstyle=\it\color[RGB]{0,96,96}\fontspec{Courier Bold},
    stringstyle=\rmfamily\slshape\color[RGB]{128,0,0}\fontspec{Courier Bold},
    showstringspaces=false,
    language=fortran,
}

\def\file#1#2{\href{#1/#2}{\underline{#2}}}

\title{2019年春季思考题报告}
\author{任俊屹,曹炅宣,张炜}
\date{}

\begin{document}

\maketitle

\section{程序思路}
    该问题为二阶其次微分方程的边界问题,只包含一个自变量。由于问题的本征值已知,可以忽略一个边界条件
    并任意指定一阶导数的初值。同时,由于变量$s$仅依赖于$\theta$,故可以将方程的自变量全部替换为$\theta$。

    为书写方便,以下记:
    \[B=\frac{\partial}{\partial s}\ln \left[h(s)^2B(s)\right]\]
    \[C=\left(2\pi f\frac{\sqrt{\mu_0}{\rho}}{B(s)}\right)\]
    \[y=\frac{\xi(s)}{h(s)}\]
    原方程化为:
    \[\frac{\partial^2}{\partial s^2}y+B\frac{\partial}{\partial s}y+Cy=0\]
    又因为:
    \[\frac{\partial}{\partial s}=\frac{1}{\frac{\partial s}{\partial\theta}}\frac{\partial}{\partial\theta}\]
    所以方程进一步化为:
    \[\frac{\partial^2}{\partial\theta^2}y+\left(-\frac{\frac{\partial^2s}{\partial\theta^2}}{\frac{\partial s}{\partial\theta}}+
      B\frac{\partial s}{\partial\theta}\right)y+C\left(\frac{\partial s}{\partial\theta}\right)^2=0\]
    这是关于$\theta$的二阶微分方程,其边界条件为:
    \[y(\theta_{m1})=y(\theta_{m2})=0\]
    $\theta_{m1}$和$\theta_{m2}$由$LR_E\sin^2\theta=R_E$确定。

    由于本征值已知,取初值条件
    \[y(\theta_{m1})=0, \quad \frac{\partial}{\partial\theta}y(\theta_{m2})=10^{-7}\]
    代替边界条件计算。

    至此方程已化为初值问题,考虑使用runge-kutta方法求解。为书写简便,记:
    \[y'=\frac{\partial}{\partial\theta}y, \quad y''=\frac{\partial^2}{\partial\theta^2}y\]
    并记$y'$项系数为$B'$,$y$项系数为$C'$。做代换:
    \[z=y'\]
    则方程化为:
    \begin{equation*}
        \begin{cases}
            y'=z \\
            z'=-B'z-C'y
        \end{cases}
    \end{equation*}
    该式符合runge-kutta方法的格式。

\section{源码文件组织}
    \subsection{Fortran源码(目录src)}
        \file{src}{parameters.f90}定义了与该问题有关的一些常数取值,\file{src}{functions.f90}定义了问题中给出的所有函数关系。
        \file{src}{runge\_kutta.f90}为之前作业中编写 的四阶runge-kutta法程序。\file{src}{equation.f90}定义了前述推导中使用的简写
        及函数。\file{src}{solve\_global.f90}定义了与方程求解过程相关的常数,\file{src}{solve.f90}为主程序,计算并输出文件。
    \subsection{临时脚本(目录scripts)}
        由于$\frac{\partial s}{\partial\theta}$、$\frac{\partial^2s}{\partial\theta^2}$和$B$的表达式十分复杂但仍为初等函数,故使用sympy
        符号计算库求出其表达式。见脚本\file{scripts}{expr.py}。该脚本将上述表达式求出并输出到txt文件,在编写程序时可以直接读入。该脚本
        及其输出文件仅用于程序编写而非程序运行。
    \subsection{其他文件}
        \file{.}{plot.py}利用输出结果绘制图片,\href{./Makefile}{\underline{Makefile}}帮助自动化编译运行。

\section{运行结果}
    对应各$f$本征值,得到的结果如下:\\
    \begin{figure}[hb]
        \centering
        \includegraphics[scale=0.6]{test.eps}
        \caption{\small 各$f$下$\xi-\theta$关系}
    \end{figure}

\end{document}
