\documentclass{ctexart}

\usepackage{geometry}
\geometry{a4paper, centering, scale=0.8}
\usepackage{listings}
\usepackage{fontspec}
\usepackage{xcolor}
\lstset{
    basicstyle=\fontspec{Courier New Bold},
    columns=fixed,
    numbers=left,
    frame=none,
    backgroundcolor=\color[RGB]{200,200,200},
    keywordstyle=\color[RGB]{40,40,255},
    numberstyle=\footnotesize\color{darkgray},
    commentstyle=\it\color[RGB]{0,96,96},
    stringstyle=\rmfamily\slshape\color[RGB]{128,0,0}\fontspec{Courier New Bold},
    showstringspaces=false,
    language=fortran,
}


\homework{五}

\begin{document}
\maketitle

\begin{answer}

    \question{1}
    用interface定义出一个新的虚拟函数Area,当调用area只输入一个浮点数时,把他当成是圆的半径值,计算并返回圆的面积。
    当输入两个浮点数时,把他们当成是矩形的两个边长,并返回矩形的面积。\\

    直接计算。测试程序输出area(5.)和area(3., 4.)的值。代码如下:
    \code{area.f90}
    测试代码如下:
    \code{test.f90}
    无输入文件,输出文件为\file{output.txt}

    \question{2}
    用 Lax-Wendroff格式编写程序求解Burgers方程,并分析CFL(柯朗数)对计算的影响。\\

    对于非线性方程
    \[\frac{\partial u(x,t)}{\partial t}+\frac{\partial f(u(x,t))}{\partial x} = 0\]
    Lax-Wendroff格式的表达式为:
    \[u_i^{n+1}=u_i^n-\frac{\Delta t}{2\Delta x}[f(u_{i+1}^n)-f(u_{i-1}^n)]+
                \frac{\Delta t^2}{2\Delta x^2}[A_{i+1/2}(f(u_{i+1}^n)-f(u_i^n))
                -A_{i-1/2}(f(u_i^n)-f(u_{i-1}^n))]\]
    式中,$A_{i\pm 1/2}$的值为$\frac12(u_i^n+u_{i\pm 1}^n)$。本例中,
    \[f(u)=\frac12u^2\]
    全局定义代码:
    \code{global.f90}
    计算代码:
    \code{routines.f90}
    非并行测试代码:
    \code{main.f90}
    并行测试代码,包括$s=\pm 1$:
    \code{parallel.f90}
    绘图代码:
    \codes{python}{draw.py}
    其中,并行测试部分的第一个输出与非并行部分相同,以下插图不再重复。
    $s = 1, CFL=0.1$:
    \pict{parallel0.eps}\\
    $s = -1, CFL=0.1$:
    \pict{parallel1.eps}\\
    $s = 1, CFL=0.5$:
    \pict{parallel2.eps}\\
    $s = 1, CFL=2$:
    \pict{parallel3.eps}\\
    非并行程序的输出文件为\file{LW.dat},相应图像为\file{LW.eps}。
    并行程序的输出文件为\file{parallel0.dat}和\file{parallel1.dat}。

    CFL较大时,会导致计算结果不稳定。

    \question{3}
    用蛙跳格式编写程序求解扩散方程,并分析CFL(柯朗数)对计算的影响。\\

    套入格式直接计算即可。由于该隐式格式简单,可以直接转换为显示格式计算,即:
    \[u_i^{n+1}=(u_i^{n-1}+\frac{2b\Delta t}{\Delta x^2}(u_{i+1}^n-u_i^{n-1}+u_{i-1}^n))/(1+\frac{2b\Delta t}{\Delta x^2})\]
    全局定义代码:
    \code{global.f90}
    计算代码:
    \code{routines.f90}
    非并行测试代码:
    \code{main.f90}
    并行测试代码,包括$s=\pm 1$:
    \code{parallel.f90}
    绘图代码:
    \codes{python}{draw.py}
    其中,并行测试部分的第一个输出与非并行部分相同,以下插图不再重复。\\
    单峰:
    \pict{parallel0.eps}\\
    双峰:
    \pict{parallel1.eps}\\
    单峰, CFL=0.5:
    \pict{parallel2.eps}\\
    单峰, CFL=10:
    \pict{parallel3.eps}\\
    非并行程序的输出文件为\file{data.dat},相应图像为\file{data.eps}。
    并行程序的输出文件为\file{parallel0.dat}和\file{parallel1.dat}。

    CFL较大时,会导致计算结果不稳定。

    \question{4}
    求$\int_0^1 \sin x dx$\\

    用梯形法,表示为数组形式,代码如下:
    \code{4.f90}
    输入文件为\file{input.txt},输出文件为\file{output.txt}。

\end{answer}
\end{document}
