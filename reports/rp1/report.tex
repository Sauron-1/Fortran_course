\documentclass{ctexart}

\usepackage{geometry}
\geometry{a4paper, centering, scale=0.8}
\usepackage{listings}
\usepackage{fontspec}
\usepackage{xcolor}
\lstset{
    basicstyle=\fontspec{Courier New Bold},
    columns=fixed,
    numbers=left,
    frame=none,
    backgroundcolor=\color[RGB]{200,200,200},
    keywordstyle=\color[RGB]{40,40,255},
    numberstyle=\footnotesize\color{darkgray},
    commentstyle=\it\color[RGB]{0,96,96},
    stringstyle=\rmfamily\slshape\color[RGB]{128,0,0}\fontspec{Courier New Bold},
    showstringspaces=false,
    language=fortran,
}

\begin{document}

\section{第一次作业}
    
    保存文件夹为hw1。

    \begin{enumerate}

        \item
            请写出如下三组数对应的二进制、八进制和十六进制形式。\\
            B'110010'\\
            O'62'\\
            Z'32'\\

            对于$l$位$n$进制数,其值为
            \[\sum_{i=0}^{l-1} kn^i\]
            其中,$k$为第$i+1$位上的数。\\
            计算得,上述值的十进制表示均为$50$。

        \item
            请采用Fortran90 自由格式编写程序,计算 3**2**3=?\\

            直接打印,代码如下:
            \code{1}{2}

            目录为2,无输入文件,输出文件为2.txt,结果为6561。(编译环境:mingw-win64, i686, 8.1.0, gfortran,下同)

        \item
            请采用Fortran90 自由格式编写程序,检验逻辑型变量能否直接参与数学计算。已知逻辑型变量A=.ture.,请输出A-1,A+0, A+1 和 A+2 的具体值。\\

            代码如下:
            \code{1}{3}

            目录为3,由以下命令编译:\\gfortran 3.f90 -o 3.exe 2>3.txt\\
            得到3.txt,编译未通过,提示对逻辑量不能进行加减运算。

        \item
            请采用Fortran90 自由格式编写程序,求解实系数一元二次方程ax2+bx+c=0的解。要求:系数a(非0),b,c 为程序执行时键盘输入;所有的解(包括实数和复数)都输出到屏幕。\\

            利用自动类型转化,直接进行复数运算。代码如下:
            \code{1}{4}

            目录为4,输入文件为4\_in.txt,输出文件为4\_out.txt。

    \end{enumerate}

\section{第二次作业}

    保存文件夹为hw2。
    
    \begin{enumerate}

        \item
            如所得税有3个等级,月收入在1000元以下的税收为3\%,在1000至5000元之间的税率为10\%,在5000元以上的税率为15\%。
            请写一个程序来输入一位上班族的月收入,并计算他(她)所应缴纳的税金。\\

            假设税收为分段,及前1000元按3\%收税,1000到5000元部分按10\%收税,以此类推,代码如下:
            \code{2}{1}

            目录为1,输入文件为1\_in.txt,输出文件为1\_out.txt。

        \item{4}
            请采用IF … GOTO …语句编写Fortran90 自由格式程序,求1+2+3+…+100=?\\

            代码如下:
            \code{2}{4}

            目录为4,无输入文件,输出文件为4.txt。

    \end{enumerate}

\end{document}
