\documentclass{ctexart}

\usepackage{geometry}
\geometry{a4paper, centering, scale=0.8}
\usepackage{listings}
\usepackage{fontspec}
\usepackage{xcolor}
\lstset{
    basicstyle=\fontspec{Courier New Bold},
    columns=fixed,
    numbers=left,
    frame=none,
    backgroundcolor=\color[RGB]{200,200,200},
    keywordstyle=\color[RGB]{40,40,255},
    numberstyle=\footnotesize\color{darkgray},
    commentstyle=\it\color[RGB]{0,96,96},
    stringstyle=\rmfamily\slshape\color[RGB]{128,0,0}\fontspec{Courier New Bold},
    showstringspaces=false,
    language=fortran,
}


\homework{九}

\begin{document}

\maketitle

\begin{answer}
    \question{1}
    画写轮眼。\\

    由于所涉及的图案中图形多为几个简单图形的变换,考虑先实现有该功能的类,
    以方便使用。用二维数组表示一个图形的各个顶点,则缩放和旋转变换均为一个
    矩阵与该数组相乘,平移变换为x、y两行加平移量。

    同时,观察到图形多由直线与弧线组成,编写函数生成给定两点间绘制给定参数
    弧线和直线上的数组,具体见代码。

    由于本次代码较长,不再直接写入报告,仅列出代码文件链接。

    工具代码:\file{util.py};绘图代码:\file{test.py}。\\
    绘图效果如下:\\
    \pict{test.eps}\\
    无标准流输入、输出文件。
\end{answer}
\end{document}
