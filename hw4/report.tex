\documentclass{ctexart}

\usepackage{geometry}
\geometry{a4paper, centering, scale=0.8}
\usepackage{listings}
\usepackage{fontspec}
\usepackage{xcolor}
\lstset{
    basicstyle=\fontspec{Courier New Bold},
    columns=fixed,
    numbers=left,
    frame=none,
    backgroundcolor=\color[RGB]{200,200,200},
    keywordstyle=\color[RGB]{40,40,255},
    numberstyle=\footnotesize\color{darkgray},
    commentstyle=\it\color[RGB]{0,96,96},
    stringstyle=\rmfamily\slshape\color[RGB]{128,0,0}\fontspec{Courier New Bold},
    showstringspaces=false,
    language=fortran,
}


\homework{四}

\begin{document}

\maketitle

    \begin{answer}

    \question{1}
        求50以内的勾股数,即求满足A的平方加B的平方等于C的平方的A、B、C。要求: \\
        1. A、B和C均小于或等于50,且没有重复解。例如,A=3,B=4,C=5及A=4,B=3,C=5即是重复解。\\
        2. 格式化输出结果,例如 $3^\land2+4^\land2=5^\land2$ \\

        直接暴力求解:在保证$a<b<c$的情况下,遍历50以内所有值,并输出其中的勾股数。代码如下:
        \code{1.f90}
        无输入文件,输出文件为\file{output.txt}。

    \question{2}
        利用随机数函数产生若干个整数放在一个无格式顺序文件中,然后删除其中的奇数。\\

        生成并存放随机数,然后将其中的偶数输出到另一个文件中。代码如下:
        \code{2.f90}
        原始随机数存储在\file{origin.dat}中,删除奇数后的存放在\file{no\_odd.dat}中。两文件均非文本文件。

    \question{3}
        文件中的数据已经按照从小到大的顺序排列,一个记录中放一个整数,请将任意一个数插入到文件中去,插入后文件中的数仍然有序。\\

        建立一个数组作为缓冲区,将文件读入缓冲区时插入该整数,当文件读取结束后将缓冲区从头写入文件。代码如下:
        \code{3.f90}
        输入文件为\file{input.txt},插入前的文件副本为\file{numbers\_origin.txt},插入后的文件为\file{numbers.txt}。

    \question{4}
        检验课件中的代码。\\

        写文件代码:
        \code{write.f90}
        写入文件\file{FILE8-4.TXT}。

        读文件代码:
        \code{read.f90}
        输出到\file{output.txt}。\\
        从输出结果中看到,变量value未被读入。可能原因是文件中第一行只有一个数,被number记录,输入格式无法找到f8.3。
    
    \end{answer}

\end{document}
