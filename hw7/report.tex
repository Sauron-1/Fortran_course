\documentclass{ctexart}

\usepackage{geometry}
\geometry{a4paper, centering, scale=0.8}
\usepackage{listings}
\usepackage{fontspec}
\usepackage{xcolor}
\lstset{
    basicstyle=\fontspec{Courier New Bold},
    columns=fixed,
    numbers=left,
    frame=none,
    backgroundcolor=\color[RGB]{200,200,200},
    keywordstyle=\color[RGB]{40,40,255},
    numberstyle=\footnotesize\color{darkgray},
    commentstyle=\it\color[RGB]{0,96,96},
    stringstyle=\rmfamily\slshape\color[RGB]{128,0,0}\fontspec{Courier New Bold},
    showstringspaces=false,
    language=fortran,
}


\homework{七}

\begin{document}

    \maketitle

    \begin{answer}

        \question{1}
        自己写一个图形符号圆圈,并用该符号画图(数据随机生成即可)。\\

        生成0到$2\pi$之间表示角度的数组,并用其计算出多个顶点。其围成的多边形在边数足够多时即可作为圆。代码如下:
        \codes{python}{circle\string_marker.py}
        生成图片如下:
        \pict{circle-marker.eps}\\
        无输入或输出文件。

        \question{2}
        编写和查找外部程序实现画图展示双方向的误差棒图。\\

        在每个坐标点处另外画出作为误差棒的线段即可。代码如下:
        \codes{python}{errorbar.py}
        matplotlib.pyplot.errorbar支持双向误差棒,可直接调用。
        \codes{python}{test.py}
        调用自编误差棒函数效果如下:
        \pict{my.eps}\\
        调用matplotlib函数效果如下:
        \pict{lib.eps}\\
        自编函数未画出误差棒端点的短线。本程序无标准流输入、输出文件。

        \question{3}
        读入数据,等值线制图展示$D_{xx}$在$\alpha-E_k$平面内的分布。

        数据格式参照老师课件中的程序读入。用matplotlib.pyplot.pcolormesh做图。(李星宇说画这个,可是等值线不是不填色的那种吗?)
        代码如下:
        \codes{python}{test.py}
        图如下:
        \pict{color.eps}\\
        无标准流上的输入、输出。

    \end{answer}

\end{document}
