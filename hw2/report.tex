\documentclass{ctexart}

\usepackage{geometry}
\geometry{a4paper, centering, scale=0.8}
\usepackage{listings}
\usepackage{fontspec}
\usepackage{xcolor}
\lstset{
    basicstyle=\fontspec{Courier New Bold},
    columns=fixed,
    numbers=left,
    frame=none,
    backgroundcolor=\color[RGB]{200,200,200},
    keywordstyle=\color[RGB]{40,40,255},
    numberstyle=\footnotesize\color{darkgray},
    commentstyle=\it\color[RGB]{0,96,96},
    stringstyle=\rmfamily\slshape\color[RGB]{128,0,0}\fontspec{Courier New Bold},
    showstringspaces=false,
    language=fortran,
}


\homework{二}

\begin{document}

\maketitle

\begin{answer}

    \question{1}
    编写程序将100到200 之间的全部偶数分解为两个质数之和,并将结果逐一输出,输出格式为 X=a+b\\

    先生成质数列表,然后对每个100到200内的整数遍历该列表,判断该数与列表中当前项的差是否仍在列表中。如是,则打印并退出遍历。代码如下:
    \code{1.f90}
    输出文件为\file{1.txt},无输入文件。

    \question{2}
    请编写Fortran 90程序完成一个二维函数分布的求值Z(x,y)=x2+y2,X取值范围为【-50,50】,Y的取值范围为【-100,100】,x和y的取值间隔为1。\\

    先生成对应于$x$,$y$的数组,在计算时利用spread函数将两个数组扩展成结果矩阵的形状,本例中其行数与$y$的长度相等。代码如下:
    \code{2.f90}
    输出文件为\file{2.txt},无输入文件。

    \question{3}
    输入若干学生的学号和四门课的成绩,求(1)全体学生的平均分;(2)把成绩高于平均分的学生学号和成绩打印出来。要求尽可能地使用数组运算,尽可能避免循环操作。

    假设计算的平均分为所有四门成绩的总平均,且

\end{answer}

\end{document}
