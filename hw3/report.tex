\documentclass{ctexart}

\usepackage{geometry}
\geometry{a4paper, centering, scale=0.8}
\usepackage{listings}
\usepackage{fontspec}
\usepackage{xcolor}
\lstset{
    basicstyle=\fontspec{Courier New Bold},
    columns=fixed,
    numbers=left,
    frame=none,
    backgroundcolor=\color[RGB]{200,200,200},
    keywordstyle=\color[RGB]{40,40,255},
    numberstyle=\footnotesize\color{darkgray},
    commentstyle=\it\color[RGB]{0,96,96},
    stringstyle=\rmfamily\slshape\color[RGB]{128,0,0}\fontspec{Courier New Bold},
    showstringspaces=false,
    language=fortran,
}

\homework{三}

\begin{document}

\maketitle

\begin{answer}
    
    \question{1}
        设计一个函数子程序将一个数字型字符转化为一个与其相同的数值型数据,
        并调用该子程序完成:\\
        (1)将字符型数据’23456’ 转化为整型数据23456 \\
        (2)将字符型数据 ’75.8’转化为实型数据 75.8.\\
        
        依次自高到低读取字符,转换成相应整数,并加入到计数器中。每读一位,计数器乘十。
        当读到小数点后,每读一位,小数计位乘十。最终,计数器中的值除以小数计位中的值即可。
        代码如下:
        \code{1.f90}
        无输入文件,输出文件为\file{out.txt}。

    \question{2}
        对于任意的二维数组A(m,n),设计一个子例行程序max(A,B,m,n,k)。其中A 是一个二维数组,m、n分别是A的行数和列数,B是一个一维数组。子程序的功能是:当参 数k=1时,求A的每列上的最大元素并存放到B(1)、B(2)...B(n)中;当参数k=2时,求A每 行上的最大元素并存放到B(1)、B(2).....B(m)中。\\

        直接调用内置函数即可。代码如下:
        \code{2.f90}
        输入文件为\file{in.txt},输出文件为\file{out.txt},输入提示信息打印在错误流,见\file{err.txt}。

    \question{3}
        设计一个子例行程序,将一个字符型数据翻译成密文,翻译规则是:\\
(1)当对应字符是一个英文字母时,将A—>Z(a—>z),B—>Y,C—>X......\\
(2)当对应字符非英文字母时,保留该字符原文。如将字符型数据’word!’应翻译 为’dliw!’\\
        
        遍历分类处理。代码如下:
        \code{3.f90}
        输入文件为\file{in.txt},输出文件为\file{out.txt},输入提示信息打印在错误流,见\file{err.txt}。
\end{answer}
\end{document}
